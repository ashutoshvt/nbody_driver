% This file should document the underlying equations used in the driver

% Document and fonts
\documentclass[a4paper,12pt]{article}
\usepackage{amsmath}
%\usepackage{mathrsfsh}

% Spacing and Sizing
\usepackage[margin=1in]{geometry}
\usepackage{setspace}
\AtBeginDocument{\doublespacing}

% Bibliography
\usepackage{rsc}

% Lists
%\usepackage{enumitem}

\title{Notes and Equations for Many Body Theory and Counterpoise Corrections}
\begin{document}
\maketitle

% Basic Many Body Theory %
\section{Basic Many Body Expansion (MBE)}
    The many body expansion is a simple expansion of a many-body system's properties (like energy) into a sum of contributions from individual bodies (monomers) and the interactions between them (\textit{n}-body terms). If all interactions are considered for an N-body system (terms n = 1, ... , N are included) then the method is exact.\cite{Kaplan1986} Any \textit{k}-body interaction term is simply the contribution of the \textit{k}-mer minus all lower-body interactions and the monomer contributions. Approximations are made by only considering up to \textit{k}-body terms, where $k<N$. A property like total energy ($E_{IJK...N}$) can then be written as the sum of the monomer contributions ($E_{i}$) with all desired \textit{n}-body interaction terms($\epsilon^{(n)}$): 
        \begin{subequations}
            \begin{equation} \label{eq:mb_energy}
                E_{IJK...N} = \sum_{i=1}^{N}E_{i} + \sum_{i=1}^{N}\sum_{j>i}^{N}\epsilon_{ij}^{(2)} + \sum_{i=1}^{N}\sum_{j>i}^{N}\sum_{k>j}^{N}\epsilon_{ijk}^{(3)} + ...
            \end{equation}
            \begin{equation} \label{eq:mb_energy2}
                \epsilon_{ij}^{(2)} = E_{ij} - E_{i} - E_{j}
            \end{equation}
            \begin{equation} \label{eq:mb_energy3} \begin{aligned}
                \epsilon_{ijk}^{(3)} = E_{ijk} - (\epsilon_{ij}^{(2)} + \epsilon_{ik}^{(2)} + \epsilon_{jk}^{(2)}) - (E_{i} + E_{j} + E_{k}) \\
                 = E_{ijk} - (E_{ij} - E_{ik} - E_{jk}) + (E_{i} + E_{j} + E_{k})
            \end{aligned} \end{equation}
        \end{subequations}

% Counterpoise Corrections %
\section{Counterpoise Correction Schemes}
    In order to account for the Basis Set Superposition Error (BSSE, as described in other documents) various "counterpoise correction" schemes have been implemented. A few of them are described here.
    \subsection{Boys-Bernardi Counterpoise (CP)\cite{Boys1970}}
        The CP scheme is solely a method of correcting for the BSSE inherent in the calculation of interaction energies of dimers. That being said, it is used as a starting point for all other relevant counterpoise corrections used today. For a dimer AB, the interaction energy ($\epsilon_{AB}^{(2)}$) can be calculated as the difference between the energy of the dimer ($E_{AB}$) and the energies of the monomers ($E_A$ and $E_B$). To combat BSSE, the CP scheme simply calculates all values in the dimer basis:
        \begin{equation} \label{eq:cp}
            \epsilon_{AB}^{(2)} = E_{AB}(AB) - E_{A}(AB) - E_{B}(AB)
        \end{equation}
        where the basis set used is in parenthesis. 

    \subsection{Site-Site Function Counterpoise (SSFC)\cite{Wells1983}}
        The SSFC scheme is a direct generalization of the CP scheme applied to MBE's of any order. All interaction terms are calculated in the "full cluster basis," or the basis of the \textit{N}-mer such that equation (\ref{eq:mb_energy}) becomes:
        \begin{equation} \label{eq:SSFC_many-body}
            E_{IJK...N} = \sum_{i=1}^{N}E_{i}(IJK...N) + \sum_{i=1}^{N}\sum_{j>i}^{N}\epsilon_{ij}^{(2)}(IJK...N) + \sum_{i=1}^{N}\sum_{j>i}^{N}\sum_{k>j}^{N}\epsilon_{ijk}^{(3)}(IJK...N) + ...
        \end{equation} 
    where equations (\ref{eq:mb_energy2}) and (\ref{eq:mb_energy3}) similarly include the \textit{N}-mer basis.

        
\newpage
\bibliographystyle{rsc}
\bibliography{Many-Body.bib}
\end{document}
